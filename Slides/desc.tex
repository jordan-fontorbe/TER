\section[Description]{Description générale du logiciel}

\subsection{Cahier des charges}


\begin{frame}

 \frametitle{Cahier des charges}

%	\begin{block}{Description du logiciel attendu}

%

%		\begin{itemize}	
%			\item Ajout le module Ethernet

%			\item Création d'un client et d'un serveur VNC

%			\item Ajouter VNC à l'émulateur QEMU

%			\item Ecriture d'un simulateur d'écran E-Ink

%		\end{itemize}

%
%	\end{block}


%\end{frame}


%\begin{block}{Documentation}
%
%Lecture de la documentation\\
%
%Compréhension des différentes couches matérielles et logicielles\\
%
%Production d'un document de synthèse (\textbf{Mémoire intermédiaire})\\
%
%\end{block}


\begin{block}{Client RFB+}

Ajout du gadget USB au noyau de la liseuse\\

Connexion via SSH\\
Ajout du support E-ink à DirectFB\\

\end{block}

\begin{block}{VNC}

Création d'un client et d'un serveur VNC\\

Ajout de VNC à l'émulateur QEMU\\
\end{block}


\end{frame}


\begin{frame}

\frametitle{Cahier des charges (suite)}


\begin{block}{QEMU}

Utilisation de VNC avec QEMU\\

Écriture d'applications pour la liseuse\\
\end{block}


\begin{block}{Simulation d'écran}

Écriture d'un simulateur d'écran E-Ink\\

\end{block}


\end{frame}
