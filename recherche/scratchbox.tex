\chapter{Scratchbox}

Scratchbox est un outils permettant de faciliter la cross-compilation, et l'installation. Scratchbox utilise pour cela une machine virtuelle sous qemu.

\section{installation et configuration}

L'installation de Scratchbox est simple sous une distribution Debian : 

\subsection*{Installation}
Pour procéder a l'installation il faut suivre ces directives  :
\begin{itemize}
	\item ajout du repo  :
		il faut ajouter le paquet suivant au fichier /etc/apt/sources.list
		\begin{lstlisting}
deb http://scratchbox.org/debian apophis main
		\end{lstlisting}
	\item 
		Les paquets à installer sont :
			\begin{itemize}
				\item scratchbox-core
				\item scratchbox-libs
				\item scratchbox-devkit-cputransp
				\item scratchbox-toolchain-arm-gcc3.4-uclibc0.9.28
			\end{itemize}
	\item Authorisation du compte utilisateur pour la connection sur les machines virtuelles : 
	\begin{lstlisting}
sudo /scratchbox/sbin/sbox_adduser <your_username>
	\end{lstlisting}	
\end{itemize}

\subsection*{Configuration de la machine virtuelle}

Une fois Scratchbox installé il faut créer la machine virtuelle pour effectuer la cross-compilation.

Pour cela il faut se connecter a la console scratchbox : 
\begin{lstlisting}
	/scratchbox/login
\end{lstlisting}
Ensuite accéder au menu de gestion des machines virtuelles : 
\begin{lstlisting}
	sb-menu
\end{lstlisting}
Il faut maintenant créer une machine virtuelle ayant la même architecture que la liseuse. 
Pour cela il suffit de suivre cette configuration : 
\begin{itemize}
	\item compiler : arm-linux-cs2009q1-203sb1
	\item architecture : arm
	\item sub-architecture : arm
%	\item C-library : glibc
	\item Devkits : cputransp
	\item cpu-transparency :  qemu-arm-0.8.2-sb2
	\item rootstrap : répondre non
	\item intall files : répondre oui et installer :
		\begin{itemize}
			\item C-library
			\item /etc
			\item Devkits 
			\item Debug links
			\item fakeroot
		\end{itemize}
	\item selectionner la machine virtuelle créée
\end{itemize}

Les fichiers de la machine virtuelle sont monter dans le système de l’hôte dans : 
/scratchbox/users/$<$user$>$/home/$<$user$>$.

\section{Cross-commilation de DirectFB}

\subsection{Dépendance de DirectFB}
DirectFB a en dépendance plusieurs librairies : 
	\begin{itemize}
		\item freetype (ici : 2.4.11)
		\item libjpeg (ici : 6b)
		\item zlib (ici : 1.2.5)
		\item libpng (ici : 1.5.14)
	\end{itemize}

\subsubsection{Compilation des dépendances}

La compilation des dépendances sous scratchbox est grandement simplifié, il suffit en effet de faire les classiques commandes : 
	\begin{lstlisting}
./configure && make && make install
	\end{lstlisting}

\subsection{Compilation de DirectFB}

La compilation de DirectFB à besoin de quelque configuration supplémentaire dans le cas de la liseuse il est en effet inutile de compiler les drivers graphique et les drivers d'entrées/sorties car la liseuse ne dispose pas de carte graphique, et les entrées/sorties standard ne sont pas présente.
Pour cela il suffit de modifier l'appel du script de configuration : 
\begin{lstlisting}
./configure --disable-x11 --with-gfxdrivers=none 
	--with-inputdrivers=none
\end{lstlisting}
~\\
La configuration étant faite il faut juste par la suite faire un make  et un make install.

\subsection{Compilation d'un exemple de test DirectFB}

Maintenant la librairie DirectFB étant installé sur la machine virtuelle, on peut compiler un exemple de test directement dans la machine virtuelle.

Pour compiler un exemple utilisant DirectFB il suffit de linker la librairie et d'indiquer l'emplacement des includes de la librairies : 
\begin{lstlisting}
	gcc -o test test.c -ldirectfb -I/usr/local/include/directfb/
\end{lstlisting}