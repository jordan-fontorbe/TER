\chapter{Explication de l'application} % a changer 

%application : 
%• les ioctl
%∘ la structure upd_data
%∘ les commandes ioctls
%• directfb
%∘ description des structures
%‣ idrectfb => structures  principale
%‣ idirectfbsurface

\section{La mise a jour du framebuffer via ioctl}

Le premier test effectué a été d'utiliser le driver pour modifier directement l'affichage.
Étant donné que les modifications du framebuffer se font directement dans le code, ce dernier est juste défini a un niveau de gris précis. La mise a jour effective du framebuffer se fait via le driver directement.

\subsection{Les commandes ioctls}

Le driver epdc fournit des commandes ioctls pour la mis a jour du buffer, ces commandes 
permettent de récupérer toutes les variables du driver (par exemple la température de l'écran), mais aussi de lancer une mise à jour de l'écran et c'est ce qui nous intéresse ici.

La fonction ioctl générique prenant cette commande sous la forme d'un entier l'intégralité des commandes utilisé ici sont disponible dans le fichier fbutils.h. %todo : a mettre en annex ?


La commandes utilisé ici est la commande de mise à jour de l'écran : 
\begin{lstlisting}
MSXC_SEND_UPDATE
\end{lstlisting}
Cette commande prend en paramètre l'adresse d'une structure mxcfb_update_data.

\subsection{la structure mxcfb_update_data}
Cette structure permet de passer l'intégralité des paramètres nécessaire au driver.
Voici les options importante a retenir : 
\begin{itemize}
	\item update_region 
		permet de définir le rectangle de mis a jour de l'écran à partir de :
		\begin{itemize}
			\item les coordonnées du coin supérieur gauche
			\item la largeur et la hauteur de la zone
		\end{itemize}
	\item alt_buffer_data
		permet de définir un buffer alloué localement dans l'espace utilisateur, 
		cette option est inutilisé ici
		
	\item update_mode : 
		définit le mode de mis a jour peut être soit :
		\begin{itemize}
			\item UPDATE_MODE_PARTIAL (ne met a jour que la région concernée)
			\item UPDATE_MODE_FULL (met a jour l'écran entier)
		\end{itemize}
	\item flags : 
		ce champs permet de savoir si on souhaite utiliser ou non le champ alt_buffer_data
	\item update_marker
		permet d'identifier une demande de mise à jour pour la synchronisation
	\item waveform_mode 
		permet de définir le mode de waveforme utilisée, définit dans la structure
		mxcfb_waveform_modes , ici on utilise uniquement le mode par défaut a cause du risque 
		d'écraser les waveformes
	\item temp
		permet de définir la température utilisé pour paramétrer la waveforme a utiliser.
	
\end{itemize}


\section{La mis a jour du framebuffer via DirectFB}
\subsection{Les principales structures}
\subsection{Utilisation des ioctls avec directfb}