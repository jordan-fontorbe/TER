\chapter{Explication de l'application} % a changer 

%application : 
%• les ioctl
%∘ la structure upd_data
%∘ les commandes ioctls
%• directfb
%∘ description des structures
%‣ idrectfb => structures  principale
%‣ idirectfbsurface

\section{La mise a jour du framebuffer via ioctl}

Le premier test effectué a été d'utiliser le driver pour modifier directement l'affichage.
Étant donné que les modifications du framebuffer se font directement dans le code, ce dernier est juste défini a un niveau de gris précis. La mise a jour effective du framebuffer se fait via le driver directement.

\subsection{Les commandes ioctls}

Le driver epdc fournit des commandes ioctls pour la mis a jour du buffer, ces commandes 
permettent de récupérer toutes les variables du driver (par exemple la température de l'écran), mais aussi de lancer une mise à jour de l'écran et c'est ce qui nous intéresse ici.

La fonction ioctl générique prenant cette commande sous la forme d'un entier l'intégralité des commandes utilisé ici sont disponible dans le fichier fbutils.h. %todo : a mettre en annex ?


La commandes utilisé ici est la commande de mise à jour de l'écran : 
\begin{lstlisting}
	MXC_SEND_UPDATE
\end{lstlisting}
Cette commande prend en paramètre l'adresse d'une structure mxcfb_update_data.

\subsection{la structure mxcfb_update_data}
Cette structure permet de passer l'intégralité des paramètres nécessaire au driver.
Voici les options importante a retenir : 
\begin{itemize}
	\renewcommand{\labelitemi}{$\bullet$}
	\item update_region \\
		permet de définir le rectangle de mis a jour de l'écran à partir de :
		\begin{itemize}
			\item les coordonnées du coin supérieur gauche
			\item la largeur et la hauteur de la zone
		\end{itemize}
	\item alt_buffer_data\\
		permet de définir un buffer alloué localement dans l'espace utilisateur, 
		cette option est inutilisé ici
		
	\item update_mode \\
		définit le mode de mis a jour peut être soit :
		\begin{itemize}
			\item UPDATE_MODE_PARTIAL (ne met a jour que la région concernée)
			\item UPDATE_MODE_FULL (met a jour l'écran entier)
		\end{itemize}
	\item flags\\
		ce champs permet de savoir si on souhaite utiliser ou non le champ alt_buffer_data
	\item update_marker\\
		permet d'identifier une demande de mise à jour pour la synchronisation
	\item waveform_mode \\
		permet de définir le mode de waveforme utilisée, définit dans la structure
		mxcfb_waveform_modes , ici on utilise uniquement le mode par défaut a cause du risque 
		d'écraser les waveformes
	\item temp\\
		permet de définir la température utilisé pour paramétrer la waveforme a utiliser.
	
\end{itemize}

\subsection{L'envoie des information sur le framebuffer}

Comme on utilise pas le champs alt_buffer_data il faut mettre à jour directement le framebuffer, via le fichier special /dev/fb0, pour cela il suffit juste de créer dans le programme un espace mémoire de la bonne taille pour le mapper sur l'emplacement de ce fichier.

Ensuite un appel à l'ioctl MSXC_SEND_UPDATE, permet de mettre à jour l'affichage directement.

\section{La mis a jour du framebuffer via DirectFB}

DirectFB fournissant les primitives d'affichage il est possible de faire des choses un peu plus
évolué que de colorié le framebuffer dans une couleur donnée.
L'exemple de test dessine ainsi en boucle 10 lignes avec points de départ fixe mais points d'arrivée aléatoire.

\subsection{Les principales structures}

Toute la librairie DirectFB repose sur la structure IDirectFB. Cette dernière permet la génération de toutes les autres structures.~\\
 Le programmes de test utilise principalement les structures :
\begin{itemize}
	\item DFBSurfaceDescription qui permet de décrire la surface d'affichage
	\item IDirectFBSurface qui représente la surface d'affichage
\end{itemize}

\subsubsection{Paramètre de l'affichage}

%a voir si on laisse
Sur la liseuse l'affichage dispose d'une résolution de 800*600 pixels et le format des pixels 
réglé par default est le RGBA16, c'est paramètre sont détecté automatiquement par DirectFB.

La structure DFBSurfaceDescription dispose d'un champs caps décrivant les capacités de la surface d'affichage, ici les paramètres sont : 
	\begin{itemize}
		\item DCAPS_FLIPPING
			indique que l'affichage a besoin d'un appel à Flip pour être mis a jour.
		\item DCAPS_PRIMARY
			indique que la surface est la surface d'affichage principale
	\end{itemize}

La création de la surface d'affichage se fait via : 
	\begin{lstlisting}
	IDirectFB->CreateSurface( IdirectFB , DFBSurfaceDescription, 
			IDirectFBSurface)
	\end{lstlisting}

Il faut ensuite définir la résolution et la couleur de l'affichage via les appels a :
	\begin{lstlisting}
	IDirectFBSurface->GetSize( IDirectFBSurface, int width, 
			int height)
	IDirectFBSurface->SetColor(IdirectFBSurface,uint8,uint8,unint8)
	\end{lstlisting}

Une fois tous les réglages effectué on peut appeler les primitives graphique : 
\begin{itemize}
	\item FillRectangle
	\item FillRectangles
	\item DrawRectangle
	\item DrawLine
	\item DrawLines
	\item FillTriangle
	\item FillTriangles
	\item FillSpans
\end{itemize}

Pour mettre à jour l'affichage il faut faire un appel à : 
	\begin{lstlisting}
	IDirectFBSurface->Flip( IDirectFBSurface, DFBRegion, DFBSurfaceFlipFlags)
	\end{lstlisting}

%	\begin{itemize}
%		\item IDirectFBSurface \\
%		permet l’interaction avec le framebuffer, gestion des primitives graphique
%		\item IDirectFBInputDevice \\
%		permet la gestion des événements d'entrée utilisateur
%	\end{itemize}



\subsection{Utilisation des ioctls avec directfb}

Au stade actuel on a le framebuffer mis a jour par DirectFB. Cependant le driver n'actualise l'affichage que sur demande.
Il faut donc ajouter un appel a l'ioctl SEND_UPDATE vu précédemment.

Cependant un autre problème se pose. Le programme n'étant pas synchronisé avec le drivers, il envoie ces commandes d'affichages plus vite que le driver ne peut les gérer.
Le driver ne peut gérer que 12 tables LuT pour faire une mise à jour, une fois ces dernières remplit le drivers refuse la mise à jour.

Pour forcer la mise à jour il faut :
\begin{itemize}
	\item définir le champs update_marker de la structure mxcfb_update_data
	\item faire un appel ioctl MXCFB_WAIT_FOR_UPDATE_COMPLETE avec l'update_marker
\end{itemize}
