\section {Cross-compilation Dropbear(SSH)}
Version de dropbear utilisé : dropbear-2013.58
Cross-compilateur utilisé : arm-none-linux-gnueabi-gcc-4.7.2
Paramètre de PATH utilisé :

\begin{lstlisting}
export PATH=$PATH:/PATH-cross-compilateur/bin
export CC=arm-none-linux-gnueabi-gcc
export CXX=arm-none-linux-gnueabi-g++
export CPP=arm-none-linux-gnueabi-cpp
export AR=arm-none-linux-gnueabi-ar
export RANLIB=arm-none-linux-gnueabi-ranlib
export LD=arm-none-linux-gnueabi-ld
export STRIP=arm-none-linux-gnueabi-strip
export CCFLAGS="-march=armv7-a -mtune=cortex-a8 -mfpu=vfp"
\end{lstlisting}

Configuration cross-compilation Dropbear : 

\begin{lstlisting}
./configure --host=arm-none-linux-gnueabi CPPFLAGS=-I/home/arkh/Documents/TER/DFB/include LDFLAGS="-L/home/arkh/Documents/TER/DFB/lib"
\end{lstlisting}

Ensuite lancer les commandes suivantes :

\begin{lstlisting}
export STATIC=1 MULTI=1 CC=arm-none-linux-gnueabi-gcc SCPPROGRESS=0 PROGRAMS="dropbear dropbearkey scp dbclient"
make strip
\end{lstlisting}

Puis copier dropbearmulti dans le répertoire /bin de l'image de la Liseuse. Pour finir créer des liens symboliques avec les commandes suivantes (Lancer du pc hôte car l'image est en lecture seule) :

\begin{lstlisting}
ln -s dropbearmulti dropbear
ln -s dropbearmulti dropbearkey
ln -s dropbearmulti scp
ln -s dropbearmulti dbclient
\end{lstlisting}

Pour le bon fonctionnement de dropbear, il est nécessaire d'avoir un mot de passe root sur le système. pour en générer un, utiliser la commande gpasswd.

De plus, il est nécessairfe de générer les clef RSA et DSS via les commandes suivantes : 

\begin{lstlisting}
dropbearkey -t rsa -f dropbear_rsa_host_key
dropbearkey -t dss -f dropbear_dss_host_key
\end{lstlisting}

Normalement les fichiers sont générés dans le répertoire /etc/dropbear.
Pour démarrer dropbear (le serveur ssh de la Liseuse) au démarrage de l'image de la Liseuse, ajouter la ligne suivante à la fin du fichier /etc/rc.d/rc.local de l'image :

\begin{lstlisting}
dropbear start
\end{lstlisting}

Si vous voulez vous connecter en SSH au PC hôte, il faut utiliser la commande suivante : 
\begin{lstlisting}
dbclient [Nom-User]@[AdresseIP]
\end{lstlisting}