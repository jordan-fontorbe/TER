\chapter{Introduction au domaine}
Ce TER, Android au pays des liseuses, fait appel à plusieurs technologies. Dans cette partie, nous allons nous intéresser à ces différentes technologies afin de préciser les avantages et les inconvénients de chacune d'entre elles.

\section{Les écrans E-Ink}
L'E-Ink est une technologie d'écran qui permet de construire des écrans offrant de nombreux avantages.

\subsection{Avantages}
La consommation d'énergie est très faible. En effet, les écrans E-Ink ne consomment de l'énergie que pour mettre à jour l'affichage. Ainsi, ces écrans n'ont pas de rétro-éclairage ce qui apporte un confort visuel proche du papier et ne fatigue pas les yeux du lecteur. 

\subsection{Inconvénients}
Par contre, les écrans E-Ink, ont une durée de rafraîchissement très longue ce qui est pénalisant pour afficher des animations par exemple. On estime que pour rafraîchir entièrement et "proprement" un écran de 7 pouces, il faut environ 1 seconde. Ce problème peut être atténué en ne rafraîchissant qu'une partie de l'écran permettant de garder une partie de l'ancien affichage. 

\subsection{Technologies}

Plusieurs technologies existent pour les écrans E-Ink, comme les écrans à cristaux liquides bistables consistant à l'utilisation de cristal liquide ayant deux états stables sélectionnés. En utilisant un signal électrique donnant du blanc pour un état et le noir pour l'autre et quand un état stable est atteint, la consommation d'énergie est nulle et l'affichage est conservé. D'autres types d'écrans existent comme les écrans électrophorétiques ou encore les écrans polychromatiques. Nous allons nous intéresser plus particulièrement aux écrans électrophorétiques.

\newpage

\subsection{Les écrans électrophorétiques}

La liseuse à laquelle nous nous intéressons (Sony PRS-T1) possède un écran utilisant la technologie de l'électrophorétique.

L'écran possède des micro-capsules contenant des particules blanches chargées négativement ainsi que des particules noires chargées positivement. Lorsque l'on applique un champ électrique, les particules blanches et noires se séparent (les blanches vont vers une extrémité de la capsule et les noires vers l'autre). Ainsi avec plusieurs millions de capsules et en appliquant des champs électriques, on peut afficher une image en noire et blanc. Une fois que la capsule à reçu un signal électrique, celle-ci garde son état.

Cette technologie à pour principal avantage sa très faible consommation d'énergie puisqu'une fois l'affichage de l'écran modifié, il conserve son état sans consommation d'énergie et ne nécessite pas d'éclairage particulier (la lumière ambiante suffit).

\begin{center}
	\includegraphics{Electrophoretic.png}
	\caption{Schéma de fonctionnement d'un écran électrophorétique}
\end{center}


\newpage

\section{Android}
Android est un système d'exploitation open source utilisant le noyau Linux annoncé officiellement le 5 novembre 2007 par Google. Il est conçu pour des terminaux mobiles par la société Android, une start-up rachetée par Google.

\subsection{Android pour liseuses}
Pour les liseuses, Android permet une adaptation relativement simple. L'avantage principal étant le système open-source qui permet de pouvoir optimiser le framework d'Android, comme montré dans l'exemple ci-dessous pour le cas de Freescale.\\
\includegraphics[scale=0.45]{Android.png}

\newpage

\section{Handheld devices}
Les handheld devices regroupe l'ensemble des terminaux mobiles. Ce qu'il faut noter en particulier avec ces terminaux c'est le mode de fonctionnement un peu différent des ordinateurs dit 'de bureau'. En effet, par économie d'espace et d'énergie, les constructeurs ont recours aux solutions System on Chip (SoC).

\subsection{System on Chip}
Un SoC est un système complet embarqué sur une puce, pouvant comprendre de la mémoire, un ou plusieurs microprocesseurs, des périphériques d'interface, ou tout autre composant nécessaire à la réalisation de la fonction attendue.
\\On retrouve ces systèmes embarqués dans divers produits comme des téléphones, des imprimantes, des systèmes de navigation pour voiture, des tablettes numériques et bien évidemment sur des liseuses.