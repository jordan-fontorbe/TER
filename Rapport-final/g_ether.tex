\section {Utilitaire}
\subsection{Montage du système de fichier de la liseuse}
	%copie des fichiers (montage de la carte sd ...)
	mount -o loop -t ext4 [PATH_to_sd_card]/Os firmware/files/update.img  [dossier_de_montage]


\section{Compilation du module g_ether}
	%configuration de la compilation(cf lien)
	\subsection{Configuration}
		On utilise le compilateur gcc 4.4.3 fournit avec le NDK android.
		(Assurer la présence à avoir le NDK android dans son PATH).
		configuration du NDK : 
		\begin{verbatim}
			export CROSS_COMPILE=arm-eabi-
			export PATH=[PATH_to_ndk]/linux-x86/toolchain/arm-eabi-4.4.3/bin
			export SUBARCH=arm
			export ARCH=arm
		\end{verbatim}
		Mise en place de la configuration officielle du kernel : 
		\begin{verbatim}
			zcat [PATH_to_terkit]/nopid/livedata/config.gz >[PATH_to_terkit]/sony/linux-2.6.35.3/.config
		\end{verbatim}
		Ensuite il faut faire un make menuconfig et éditer  : 
			\begin{itemize}
				%certains : 
				\item  Device drivers $>$ usb support $>$ usb gadget support $>$ Ethernet gadget -$>$ $<$M$>$ with RNDIS support
				%peut-être useless
				\item Device drivers $>$ usb support $>$ usb gadget support $>$ CDC composite device
			\end{itemize}

Pour compiler : "make modules"
		
		Ensuite il faut copier les fichiers g_ether.ko et g_cdc.ko sur "/lib/modules/[version]/kernel/drivers/usb/gadget" (système de fichier de la liseuse)
	
	%installation des modules (modification des scripts)
\subsection{Installation des modules}	
	Les deux modules g_serial et g_ether ne peuvent pas être lancés en même temps, il faut donc modifier sur la liseuse le fichier "/etc/rc.d/rc.local" de manière à ce qu'il coupe le module g_serial et lance g_ether à la place.
	%ajout code rc.local
	\begin{verbatim}
		echo "modprobe -r g_serial" > /initrd/mnt/sd/test/log
		#desactivation de g_serial
		modprobe -r g_serial 2>> /initrd/mnt/sd/test/log
		echo "modprobe g_ether" >> /initrd/mnt/sd/test/log
		#activiation de g_ether
		modprobe g_ether host_addr=00:00:00:00:00:01 dev_addr=00:00:00:00:00:02 2>> /initrd/mnt/sd/test/log
		#on demarre l'interface
		ifconfig usb0 up
	\end{verbatim}
	Les options host_addr et dev_addr permettent à la liseuse d'affecter son adresse MAC mais aussi celle du pc sur lequel elle est connecté (host), cela permet de ne pas avoir à refaire la configuration réseau à chaque branchement.	
	%connection depuis le pc
	Pour se connecter depuis le pc penser à vérifier que le module usbnet est bien activé	
	Un interface réseau devrait apparaître, il faut la configurer sur le sous-réseau 192.168.1.0/24 (la liseuse est configurer sur l'adresse 192.168.1.1/24)
	%
	%le fichier log permet ... de faire des logs [captain obvious]
	%modification des scripts de demarrage
	