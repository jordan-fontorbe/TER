\section {Codes Sources nécessaire}
Toutes les actions suivantes ont été réalisé sur ARCHLINUX x64 avec kernel > 3.8
\subsection{Listes des différents codes sources à cross-compiler}

\begin{itemize}
\item DirectFB v1.6.3
\item Freetype v2.4.11
\item Jpegsrc v8c
\item Libconio v1.0.0
\item Libpng v1.5.14
\item LibVNCServer v0.9.7
\item SDL v1.2.15
\item Zlib v1.2.5
\end{itemize}

\subsection{Paramètres de cross-compilation}
Cross-compilateur utilisé : ARM GNU/Linux cross-compile toolkit téléchargé sur le site suivant : http://www.codesourcery.com/gnu_toolchains/arm/download.html
Paramètre de PATH de base utilisé pour chaque cross-compilation :

\begin{lstlisting}
export PATH=$PATH:/PATH-cross-compilateur/bin
export CC=arm-none-linux-gnueabi-gcc
export CXX=arm-none-linux-gnueabi-g++
export CPP=arm-none-linux-gnueabi-cpp
export AR=arm-none-linux-gnueabi-ar
export RANLIB=arm-none-linux-gnueabi-ranlib
export LD=arm-none-linux-gnueabi-ld
export STRIP=arm-none-linux-gnueabi-strip
export CCFLAGS="-march=armv7-a -mtune=cortex-a8 -mfpu=vfp"
\end{lstlisting}

\subsection{Cross-compilation libjpeg}
Configuration Makefile :

\begin{lstlisting}
./configure --enable-shared \
--host=arm-none-linux-gnueabi --build=i386-linux \
--target=arm-none-linux-gnueabi
\end{lstlisting}

Ensuite lancer la commande make
Puis déplacer les fichiers ./libs/libjpeg.so* vers un répertoire regroupant toutes les LIBS générées (PATH-REP-LIBS)
Pour finir déplacer tous les fichiers *.h vers un répertoire regroupant des fichier .h utilisé durant les cross-compilations future (PATH-REP-INCLUDE).

\subsection{Cross-compilation zlib}
Configuration Makefile :

\begin{lstlisting}
./configure --shared
\end{lstlisting}

Ensuite lancer la commande make
Puis déplacer les fichiers libz.so* vers un répertoire regroupant toutes les LIBS générées (PATH-REP-LIBS)

\subsection{Cross-compilation libpng}
Configuration Makefile :

\begin{lstlisting}
cp scripts/makefile.linux Makefile
modify Makefile for cross compile
ligne 23 : change "gcc" to "arm-none-linux-gnueabi-gcc"
\end{lstlisting}

Ensuite lancer la commande make
Puis déplacer les fichiers libpng15.so* vers un répertoire regroupant toutes les LIBS générées (PATH-REP-LIBS)

\subsection{Cross-compilation freetype}
Configuration Makefile :

\begin{lstlisting}
./configure --host=arm-none-linux-gnueabi
\end{lstlisting}

Ensuite lancer la commande make
Puis déplacer les fichiers objs/libs/libfreetype.so* vers un répertoire regroupant toutes les LIBS générées (PATH-REP-LIBS)
Pour finir déplacer tous les fichiers include/*.h vers un répertoire regroupant des fichier .h utilisé durant les cross-compilations future (PATH-REP-INCLUDE).

\subsection{libVNCServeur}
Nécessite quelques libs supplémentaire :

\subsubsection{libconio}
Configuration Makefile :

\begin{lstlisting}
./configure --host=arm-none-linux-gnueabi --build=i386-linux \
--target=arm-none-linux-gnueabi --enable-shared --disable-static
\end{lstlisting}
Ensuite lancer la commande make
Pour finir déplacer tous les fichiers *.h vers un répertoire regroupant des fichier .h utilisé (PATH-REP-INCLUDE)

\subsubsection{SDL}
Configuration Makefile :

\begin{lstlisting}
./configure CPPFLAGS=-I/home/arkh/Documents/TER/DFB/include \ 
LDFLAGS=-L/home/arkh/Documents/TER/DFB/lib \
--host=arm-none-linux-gnueabi --build=i386-linux \ 
--target=arm-none-linux-gnueabi \
--enable-shared --enable-static \
--disable-video-x11 --disable-pulseaudio
\end{lstlisting}

Ensuite lancer la commande make
Puis déplacer les fichiers libSDL.so* vers un répertoire regroupant toutes les LIBS générées (PATH-REP-LIBS)
Pour finir déplacer tous les fichiers *.h vers un répertoire regroupant des fichier .h utilisé (PATH-REP-INCLUDE)

\subsubsection{Cross-compilation libVNCServeur}

Configuration Makefile :

\begin{lstlisting}
./configure CPPFLAGS=-I/home/arkh/Documents/TER/DFB/include \
LDFLAGS=-L/home/arkh/Documents/TER/DFB/lib \  
	--host=arm-none-linux-gnueabi \
	--build=i386-linux \
	--target=arm-none-linux-gnueabi \ 
	--enable-shared \
	--disable-static \
	--without-ssl \
	--without-x
\end{lstlisting}

Ensuite lancer la commande make
Puis déplacer les fichiers libVNCServeur.so* vers un répertoire regroupant toutes les LIBS générées (PATH-REP-LIBS)
Copier le répertoire rfb et son contenu vers le répertoire contenant DirectFB à cross-compiler.

\subsection{Cross-compilation DirectFB}

Configuration Makefile :

\begin{lstlisting}
PKG_CONFIG_PATH=/home/arkh/Documents/TER/DFB/archive/libpng-1.5.14/ \
LIBPNG_CFLAGS=-I/home/arkh/Documents/TER/DFB/include \
LIBPNG_LDFLAGS="-L/home/arkh/Documents/TER/DFB/lib -lpng15 -lz" \
FREETYPE_CFLAGS=-I/home/arkh/Documents/TER/DFB/include \
FREETYPE_LIBS="-L/home/arkh/Documents/TER/DFB/lib -lfreetype" \
GL_LIBS="-L/usr/lib/" \
LIBS="-lgcc_s -lgcc -ldl -lstdc++ -lz -lm -L/usr/include/GLES -L/usr/include/GL" \
CXXFLAGS="-march=armv7-a" \
./configure CC=arm-none-linux-gnueabi-gcc CPPFLAGS=-I/home/arkh/Documents/TER/DFB/include LDFLAGS=-L/home/arkh/Documents/TER/DFB/lib \
   --disable-imlib2 \
   --build=i686-linux --host=arm-linux \
   --disable-static --enable-shared \
   --enable-freetype --enable-fbdev --disable-x11 \
   --disable-mesa --disable-x11vdpau \
   --with-gfxdrivers=none --with-inputdrivers=none --enable-vnc
\end{lstlisting}

Ensuite lancer la commande make

