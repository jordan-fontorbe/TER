\documentclass[12pt,a4paper,utf8x]{report}
\usepackage [frenchb]{babel}

%\usepackage{ucs}
\usepackage{pgfgantt}
\usepackage{graphicx}
\usepackage{caption}
\usepackage[utf8x]{inputenc}
\usepackage{multicol}
\usepackage{url} % Pour avoir de belles url
\usepackage {geometry}
%\usepackage {listings} %Pour mettre du code source 
\usepackage{verbatim}
\usepackage{lscape} % Pour pouvoir passer en paysage
\usepackage{geometry}
\usepackage{pdflscape}
\usepackage{colortbl}
\usepackage[strings]{underscore}
\geometry{left=2.5cm,right=2.5cm,vmargin=2cm}


%chapitre---------------------------------------------------------------------
 
%%%% debut macro pour enlever le nom chapitre %%%%
\makeatletter
\def\@makechapterhead#1{%
  \vspace*{50\p@}%
  {\parindent \z@ \raggedright \normalfont
    \interlinepenalty\@M
    \ifnum \c@secnumdepth >\m@ne
        \Huge\bfseries \thechapter\quad
    \fi
    \Huge \bfseries #1\par\nobreak
    \vskip 40\p@
  }}

\def\@makeschapterhead#1{%
  \vspace*{50\p@}%
  {\parindent \z@ \raggedright
    \normalfont
    \interlinepenalty\@M
    \Huge \bfseries  #1\par\nobreak
    \vskip 40\p@
  }}
\makeatother
%%%% fin macro %%%% 

\begin{document}

\begin{titlepage}
\begin{flushright}
   	\includegraphics[scale=0.30]{univorleans.png}\\ 
   	   	Département Informatique
\end{flushright}
\vspace{30mm}
\begin{center}
\huge{Mémoire intermédiaire \\Travaux d'étude et de recherche }\\
\vspace{8mm}
\large{Sujet : Android au pays des liseuses}\\
\vspace{3mm}
\large{Proposé et encadré par : Ollinger Nicolas}
\vspace{3mm}
\large{\\Réaliser par :}\\
\large{Fontorbe Jordan, Guillaume Arthur, Monediere Tristan, \\Rubagotti Joris}\\
\end{center}
\begin{figure}[b!]
\begin{flushright}
~~\\ ~~\\ ~~\\ ~~\\ ~~\\ ~~\\ ~~\\
\large{Année : 2012-2013}
\end{flushright}
\end{figure}
\end{titlepage}

\tableofcontents
\clearpage

\chapter{Résumé du projet}

Notre projet est proposé par M. OLLINGER et nous entraîne dans le monde des liseuses. Il consiste à étudier dans un premier temps les spécificités du déploiement d'Android sur une liseuse. Dans un second temps, d'émuler la plate-forme à l'aide d'une machine virtuelle sur un écran déporté sur un ordinateur. Tout le développement se concentrera sur la liseuse Sony PRS-T1 et son environnement Android fourni par notre chef de projet. 

%\section{Documentation}
%La première partie du projet a pour objectif de se documenter sur le sujet, c'est à dire la technologie du papier électronique et comment elle est mise en oeuvre au travers des différentes couches matérielles et logicielles.
%Cette partie va servir a la production de ce premier document de synthèse. 
%\section{Écriture du client RFB}
%Cette étape, a pour but de s'approprier les outils mis à notre disposition pour le développement. Il faudra dans un premier temps tester l'image de la liseuse. Puis mettre en oeuvre l'environnement de développement ltib de Freescale pour i.MX508. Il faut ensuite écrire un programme de test du framebuffer eink et ajouter un ioctl de mise à jour pour ensuite développer le support du gadget USB Ethernet au noyau. Il faut aussi ajouter ce module à l'image de test et une connexion via ssh. Il faut modifier DirectFB pour le support e-ink. Il faudra enfin modifier le protocole RFB pour supporter les mises à jour ioctl et mettre en oeuvre un client RFB pour la liseuse après avoir créé un serveur de gestion des mises à jour et effectuer des tests intensifs.
%\section{E-ink sous qemu via VNC}
%L'objectif de cette étape est de comprendre la gestion du framebuffer par l'émulateur qemu pour pouvoir y ajouter la gestion des ioctl. Dans un deuxième temps la compréhension de l'option VNC de qemu permettra l'ajout de l'extension RFB du client de la liseuse. Enfin des tests seront menés pour valider l'étape.
%\section{Liseuse Android sous qemu}
%Dans cette étape la compréhension des versions e-ink d'Android chez Freescale et Sony permettra la mise en place de cette pile dans l'émulateur. Cette étape sera testée à l'aide d'applications Sony. Enfin nous pourront écrire nos propres applications pour l'émulateur.
%\section{Simulateur d'écran E-ink}
%Cette dernière étape a pour objectif d'écrire un simulateur raisonnable d'écran e-ink. Celui-ci devra contenir une option de debug permettant d'afficher les zones de mise à jour. Le simulateur sera ensuite intégré à l'émulateur. Cette étape se terminera par des tests.
\chapter{Introduction au domaine}
Ce TER, Android au pays des liseuses, fait appel à plusieurs technologies. Dans cette partie, nous allons nous interesser a ces différentes technologies afin de préciser les avantages et les inconvéniants de chacunes d'entre elles.
\section{Les écrans E-ink}
L'e-ink est une technologie d'écran qui permet de construire des écrans offrant de nombreux avantages tel qu'une consomation d'énergie très limitée. En effet, les écrans e-ink ne consomment de l'énergie que pour mettre à jour l'affichage. Ainsi, ces écrans n'ont pas de retro-éclairage ce qui apporte un confort visuel proche du papier en ne fatigant pas les yeux du lecteur. Par contre, les écrans e-ink, ont une durée de rafraichissement très longue ce qui est pénalisant pour afficher des animations par exemple. On estime que pour raffraichir entièrement et "proprement" un écran de 7 pouces, il faut environ 1 seconde. Ce problème peut être limité en ne raffraichissant qu'une petite partie de l'écran ou en n'effaçant pas entièrement l'affichage précedent. Un description technique du fonctionnment des écrans à encre électronique sera donnée plus en avant dans ce rapport.
\section{}
\chapter{Analyse de l'existant}
\chapter{Besoins non fonctionnels}
\section{Hacks de la PRS-T1}

La PRS-T1 étant fermé, il nous est impossible dans sa configuration de base de pouvoir y ajouter la mise à jour 
de l'affichage depuis une machine hôte.
Plusieurs méthodes existent pour débloquer la liseuse : 

\subsubsection{Mise à jour du firmware}

La méthode la plus répandue pour débloquer la liseuse consiste à faire une mise à jour du firmware.
La liseuse n'acceptant que les mises à jour signée par SONY. Les clefs privées de cette signature étant connues des hackers (elles sont stockée dans la liseuse pour la vérification).

Plusieurs firmware modifiés existe déjà pour mettre à jour la liseuse, c'est derniers débloquent complètement la liseuse tout en gardant les logiciels SONY pré-installés.

Cette méthode dispose donc de plusieurs avantages : 
	\begin{itemize}
		\item permet un accès total a la liseuse (installation d'applications Android, ...)
		\item on conserve les fonctions normales de la liseuse (logiciel de lecture SONY)
	\end{itemize}
Cependant le fait de modifier le firmware interne de la liseuse comporte certains risques : 
	\begin{itemize}
		\item risque de rendre la liseuse inutilisable
		\item perte de la garantie constructeur
	\end{itemize}

Étant donné que la PRS-T1 est un prêt de M. Ollinger cette méthode de hack comporte trop de risques.

\subsubsection{Utilisation du mode Recovery}

Une autre méthode de hack consiste à utiliser le mode recovery de la liseuse.
Ce mode permet normalement de récupérer la liseuse suite à un problème.
Pour entrer en mode recovery la liseuse a besoin d'un système de fichier de récupération.
Ce dernier pouvant être situé sur une carte mémoire externe.

Cette méthode réduits les risques encouru lors de la manipulation car un simple redémarrage de la liseuse, 
suffit pour revenir au système de base.

Les caractéristiques de cette méthode sont  : 
%Les avantages de cette méthodes sont : 
\begin{itemize}
	\renewcommand{\labelitemi}{$\bullet$}
	\item Avantages : 
	\begin{itemize}
		\item Risque minime pour la liseuse
		\item Permet des modifications importantes (car on fait nous même le système à partir des sources du noyau fournit par SONY)%un peu long non ?
	\end{itemize}
	\item désavantages : 
		\begin{itemize}
			\item Nécessite un pc hôte pour la mise en place du système (nécessite de compiler le noyau)
			\item On repart de zéro donc on n'a pas accès aux applications SONY (pour la lecture d'Ebook notamment)
			\item Nécessite beaucoup de travail car on repart avec uniquement le noyau.
		\end{itemize}
\end{itemize}
\section{Risque}
\subsection{Les Waveforms}
	Le driver permet de redéfinir les fonctions de waveform du contrôleur.
Cependant le bon fonctionnement des écrans E-Ink dépend fortement de ces fonctions, modifier ces fonctions peut donc entraîner dans le meilleur des cas une différence entre la représentation virtuelle de l'affichage et ce que va réellement afficher l'écran, dans le pire cas cela peu endommager de manière définitive l'écran de la liseuse.

\section{Tests}

La spécificité des liseuses venant de leur écran il faut faire attention a respecter au maximum le comportement de ce dernier pour faire un environnement de développement dédié aux liseuses.

\subsubsection{Taux de rafraîchissement}

Les écrans E-Ink ayant un taux de rafraîchissement assez bas il devient nécessaire de veiller à ne pas le réduire d'avantage.
Pour cela Il va falloir mettre en place un test comparatif entre le rafraîchissement de la liseuse seule et le rafraîchissement avec un ordinateur hôte via VNC.

\subsubsection{Update bloquante}

L'ordinateur hôte ne doit pas se comporter de la même manière qu'avec un écran classique. En effet il serait très facile pour l'hôte de surcharger complètement l'écran en faisant des mises à jour de l'affichage comme il le ferait avec un écran classique. %a véifier un peu tordu

\chapter{Besoins fonctionnels}

Comme énoncé dans le chapitre 1 de ce document, le but du projet est la réalisation d'un émulateur d'une liseuse avec un écran virtuel fonctionnant comme un écran E-Ink directement sur notre machine ayant pour système d'exploitation Linux. La réalisation de ce projet se découpe en plusieurs parties à réaliser dans un ordre strict afin d'aboutir au résultat final souhaité. Ce plan nous a été proposé par notre encadrant de projet M. OLLINGER. N'ayant pas encore eu le temps de tester les points suivants il se peut que certains des éléments ont été mal interprétés d'où la présence possible d'erreurs. D'autres points peuvent aussi manquer de détails.

\section{Conception d'un client RFB pour la liseuse}

Dans un premier temps il nous faut concevoir un prototype de client permettant de mettre à jour l'écran de la liseuse de test via notre ordinateur.

\subsection{DirectFB}
Pour réaliser cela, il nous a été proposé de travailler avec DirectFB. C'est une bibliothèque libre qui fournit à la fois un accès aux composants matériels graphiques (accélération graphique) ainsi qu'aux périphériques d'entrées, et un systèmes de gestion de fenêtres intégrées avec support de la transparence et de calques multiples. Tout ceci à travers l'interface framebuffer de Linux. Nous devons ajouter la gestion des écrans E-Ink dans DirectFB. 
 
\subsection{Client RFB}
Lorsque DirectFB est prêt à être employé, on travaille ensuite sur le protocole RFB (Remote FrameBuffer) qui est un simple protocole permettant des accès à distance à des interfaces graphiques d'utilisateur. On doit ajouter la gestion des mises à jour de l'affichage pour les écrans de type E-Ink via ioctl. De tous ces éléments, on peut ainsi concevoir un client RFB interagissant sur la liseuse et la conception d'un serveur gérant les mises à jour.
\\Les tests consisteront à essayer de produire des changements d'affichage sur la liseuse grâce à des demandes précises envoyées via l'ordinateur de test et transmis par le serveur de mise à jour. 

\newpage

\section{E-Ink QEMU via VNC}

Dans un second temps, on développe la base de l'émulateur, c'est à dire la mise en place d'une machine virtuelle pouvant reproduire les actions nécessaires au fonctionnement d'un écran E-Ink. Pour réaliser cela, nous travaillons sur QEMU qui permet l'émulation de processeur et de machine virtuelle permettant l'exécution de système d'exploitation. On ajoute la gestion des ioctl à QEMU. Ensuite il faut ajouter à l'option VNC de QEMU l'extension RFB du client de la liseuse.
\\Le test consiste à vérifier si depuis une machine virtuelle QEMU utilisant l'image du noyau Linux de la liseuse de test, on peut via VNC interagir avec la liseuse. 


\section{Liseuse Android sous QEMU}

Après la création de la base de l'émulateur, on intègre à celui-ci un système d'exploitation Android conçu pour fonctionner sur une liseuse. Dans notre cas ça sera la version d'Android présente sur notre liseuse de test SONY PRS-T1.
Pour les tests, on lance l'émulateur muni d'Android et on test le fonctionnement des applications SONY fournies de base avec le système d'exploitation.

\section{Simulation d'écran E-Ink}

Pour finir le développement de l'émulateur, on développe un simulateur d'écran E-Ink reproduisant virtuellement le comportement de celui-ci en y ajoutant des fonctionnalités facilitant les tests telles que choisir une zone précise à mettre à jour sur un écran (utile au debug). Ensuite on intègre celui-ci dans notre émulateur.
\\Les tests consisteraient à réaliser des mini-applications, les lancer sur l'émulateur et vérifier les réactions de notre écran virtuel.


\chapter{Résultats de tests}

Étant donné que la PRS-T1 dispose d'un environnement initialement fermé, il nous a paru judicieux de contourner cette limitation dans un premier temps.

\section{Hack via le mode Recovery}

La première chose à faire pour débloquer la liseuse a été de tester la méthode de hack choisie.

Pré-requis : 
\begin{itemize}
	\item avoir une carte SD avec un système de fichier de récupération 
	\item avoir un terminal virtuel gérant une connexion USB en série
\end{itemize}

\subsection{Mettre la liseuse en mode Recovery}

La procédure pour mettre la liseuse en mode Recovery est assez simple, il suffit de démarrer la liseuse en appuyant sur les boutons HOME et MENU jusqu'à ce que le chargement atteigne 100\%. 

\subsection{Connexion via USB série}

Une fois la liseuse en mode Recovery le seul moyen d'y accéder se fait via un terminal.
Pour accéder au terminal sur la liseuse il faut lancer un terminal virtuel gérant une connexion USB en série, par exemple pour Linux minicom, ou Putty sur Windows.

Pour Linux : 
\begin{itemize}
	\item trouver à quel fichier a été rattaché la liseuse \\
		pour cela : 
			\begin{verbatim}
				dmesg |grep cdc_acm
			\end{verbatim}
	\item lancer et configurer minicom \\
		Lancer minicom\\
			\begin{verbatim}
				minicom -s ttyACM0
			\end{verbatim}
		Il faut ensuite changer le port série à utiliser pour la connexion, pour cela : \\	
		\begin{itemize}
			\item aller dans "configuration du port série"
			\item changer le port série par celui trouvé à l'étape précédente
		\end{itemize} \\
	\item se logger en tant que root (aucun mot de passe requis)
\end{itemize}

\section{Construction du système de fichier de Recovery}
\subsection{Compilation du noyau}

La compilation du noyau (via les sources de SONY fournit) est la première étape vers la création d'un système de fichier pour le mode Recovery.

La compilation du noyau nécessite un environnement de cross-compilation , le compilateur fourni par SONY ne fonctionnant pas nous avons utilisé celui fourni avec le NDK d'Android (arm-eabi-gcc-4.4.3).

Avant de lancer la compilation du noyau il faut définir les différents drivers et modules qui vont être utilisés. La plate-forme SONY étant particulière il faut récupérer la configuration du noyau officiel, celle-ci se trouve sur la liseuse dans le fichier : 
\begin{verbatim}
	/proc/config.gz
\end{verbatim}
 Après avoir défini les variables d'environnement pour la compilation (ARCH, SUBARCH et CROSS_COMPILE), il suffit de lancer un make.
 
 \subsection{Ajout du support d'USB gadget}
 
 Le fait que la connexion avec la liseuse se fait via un port USB série, nous retire des fonctionnalités nécessaires à l'environnement de développement. En effet VNC n'est pas prévu pour fonctionner avec un port série.
 
 Il faut donc ajouter au noyau le module USB gadget, afin d'avoir un support Ethernet via USB.

\subsection{Création du fichier}

La liseuse nécessite un fichier update.img qui est en fait une image d'un système de fichier Ext4.

Pour créer le fichier il suffit de : 
	\begin{itemize}
		\item créer un fichier de la bonne taille \\
			\begin{verbatim}
				dd if=/dev/zero of=update.img bs=1024 count=0 seek=$[1024*1024]
			\end{verbatim}
		\item construire le système de fichier \\
			\begin{verbatim}
				mkfs -t ext4 update.img
			\end{verbatim}
	\end{itemize}

\subsection{Ce qu'il reste à faire}

Cependant la création du système de fichier n'est pas encore terminée, il faut :
	\begin{itemize}
		\item créer l'arborescence nécessaire au noyau Linux
		\item faire démarrer la liseuse sur ce nouveau noyau
	\end{itemize}
\begin{landscape}
\chapter{Planning}

\begin{figure}[h!]
\begin{center}

\begin{ganttchart}[y unit title=0.4cm,
y unit chart=0.5cm,
vgrid,hgrid, 
title label anchor/.style={below=-1.6ex},
title left shift=.05,
title right shift=-.05,
title height=1,
bar/.style={fill=gray!50},
incomplete/.style={fill=cyan},
progress label text={},
bar height=0.7,
group right shift=0,
group top shift=.6,
group height=.3,
group peaks={}{}{.2}]{32}
%labels
\gantttitle{Planning}{32} \\
\gantttitle{Février}{8} 
\gantttitle{Mars}{8} 
\gantttitle{Avril}{8} 
\gantttitle{Mai}{8} \\
%tasks
\ganttbar[bar/.style={fill=green, rounded corners=3pt}]{Recherche Doc}{1}{8} \\
\ganttbar[bar/.style={fill=blue, rounded corners=3pt}]{Réunion}{9}{10} \\
\ganttbar[progress=90,bar/.style={fill=orange, rounded corners=3pt}, progress label text={Joris}]{Dev DirectFB}{11}{17} \\
\ganttbar[progress=90,bar/.style={fill=orange, rounded corners=3pt}, progress label text={Jordan}]{Dev Client RFB}{11}{17} \\
\ganttbar[progress=90,bar/.style={fill=orange, rounded corners=3pt}, progress label text={Tristan}]{Dev Serveur Update}{11}{17} \\
\ganttbar[progress=90,bar/.style={fill=orange, rounded corners=3pt}, progress label text={Arthur}]{Dev QEMU et VNC}{17}{21} \\
\ganttbar[progress=90,bar/.style={fill=orange, rounded corners=3pt}, progress label text={Jordan et Arthur}]{Dev QEMU Liseuse SONY}{18}{22} \\
\ganttbar[progress=90,bar/.style={fill=orange, rounded corners=3pt}, progress label text={Joris et Tristan}]{Dev Simulation Ecran}{23}{27} \\
\ganttbar[bar/.style={fill=yellow, rounded corners=3pt}]{Finalisation}{28}{31}
\end{ganttchart}
\vspace{-0.5cm}
\end{center}
\hspace{7.3cm} \noindent \begin{tabular}{l}
	 \rowcolor{green} \\  
\end{tabular}
Recherche documentaire

\hspace{7.3cm} \noindent \begin{tabular}{l}
     \rowcolor{blue} \\  
\end{tabular}
Réunion du groupe pour mettre en commun tous les éléments et répartir les taches 

\hspace{7.3cm} \noindent \begin{tabular}{l}
     \rowcolor{orange} \\  
\end{tabular}
Phase de développement

\hspace{7.3cm} \noindent \begin{tabular}{l}
     \rowcolor{cyan} \\  
\end{tabular}
Phase de test

\hspace{7.3cm} \noindent \begin{tabular}{l}
     \rowcolor{yellow} \\  
\end{tabular}
Finalisation du TER \\
\caption{Planning de réalisation du projet}
\end{figure}
\end{landscape}
%\include{biblio}

\bibliographystyle{plain} % Le style est mis entre accolades.
\bibliography{bibli} % mon fichier de base de données s'appelle bibli.bib
\nocite{ref1, ref2, ref3, ref4, ref5, ref6, ref7, ref8, ref9, ref10, ref11, ref12, ref13, ref14, ref15, ref16, ref17, ref18, ref19}

\end{document} 